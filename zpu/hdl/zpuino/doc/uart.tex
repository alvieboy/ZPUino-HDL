\section{UART}

ZPUino UART is a generic UART implementation, with a 16x oversampling filter. Only $8n1$ 
data format is supported (1 start bit, 8 data bits, 1 stop bit, no
parity). With a $96Mhz$ clock, baud rates up to $3Mbit$ can be used with perfect timing alignment.
\subsection{HDL sources and modules}

\subsubsection{HDL instantiation template}

\begin{lstlisting}[language=VHDL]
component zpuino_uart is
  generic (
    bits: integer := 11
  );
  port (
    wb_clk_i: in std_logic;
    wb_rst_i: in std_logic;
    wb_dat_o: out std_logic_vector(wordSize-1 downto 0);
    wb_dat_i: in std_logic_vector(wordSize-1 downto 0);
    wb_adr_i: in std_logic_vector(maxIObit downto minIObit);
    wb_we_i:  in std_logic;
    wb_cyc_i: in std_logic;
    wb_stb_i: in std_logic;
    wb_ack_o: out std_logic;
    wb_inta_o:out std_logic;

    enabled:  out std_logic;
    tx:       out std_logic;
    rx:       in std_logic
  );
end component;
\end{lstlisting}



\subsubsection{Compliance}
The UART module is wishbone compatible, in non-pipelined mode.





\subsubsection{Generics}

\begin{description}
\item{bits} \hfill \\ Number of RX FIFO address bits
\end{description}
UART FIFO size will be $2^{bits}$ bytes wide. Default FIFO address size is 11 bits ($2^{11}=2048$ bytes)

\subsubsection{Source files}

ZPUino UART is composed of the following source modules:

\begin{table}[h!]
\begin{center}
\begin{tabularx}{10cm}{ll}

zpuino\_uart.vhd & Top level UART module \\
zpuino\_uart\_rx.vhd & UART RX module \\
uart\_brgen.vhd & Baud rate generator \\
zpuino\_uart\_mv\_filter.vhd & RX Majority voting filter \\
tx\_unit.vhd & UART TX module \\
fifo.vhd & UART FIFO module

\end{tabularx}
\end{center}
\caption{UART source files}
\end{table}



\subsection{Location}

UART registers are usualy located in IOSLOT 1.

\subsection{Registers}
UART operation is controlled by three registers, $UARTCTL$, $UARTDATA$ and \\
$UARTSTATUS$.

\subsubsection{UARTCTL register}
The $UARTCTL$ register controls the transmit and receive operation of the UART.

\begin{table}[h!]
\begin{center}
\begin{tabularx}{14cm}{Xcc}
31 \hfill 17 & 16 & 15 \hfill 0  \\
\hline
\multicolumn{1}{|c|}{Reserved} & 
\multicolumn{1}{|c|}{UARTEN} & 
\multicolumn{1}{|c|}{UARTPRES} \\
\hline
\end{tabularx}
\caption{UARTCTL register}\label{UARTCTL}
\end{center}
\end{table}

\begin{description}
\item{16 - UARTEN} \hfill \\
UARTEN bit controls whether UART is enabled or not. When set to 1, UART input
and output will be mapped to appropriate pins. When set to 0 TX and RX will be
disconnected.
\item{[15:0] - UARTPRES} \hfill \\
 UART prescaler (16 bits). See \ref{UART Prescaler} for details.
\end{description}


\subsubsection{UARTDATA register}
The $UARTDATA$ register is used for tranmission and reception of UART data. Reception includes a configurable (synthesis time) FIFO.
Only the lower 8 bits of this register are used.

\begin{table}[H]
\begin{center}
\begin{tabularx}{14cm}{Xc}
31 \hfill 8 & 7 \hfill 0  \\
\hline
\multicolumn{1}{|c|}{Reserved} & 
\multicolumn{1}{|c|}{TXRXD} \\
\hline
\end{tabularx}
\caption{UARTDATA register}
\end{center}
\end{table}

\begin{description}
\item{[7:0] - TXRXD} \hfill \\
 8-bit UART data transmit and receive register.
\end{description}

\subsubsection{UARTSTATUS register}

The $UARTSTATUS$ register contains the current status of UART.


\begin{table}[H]
\begin{center}
\begin{tabularx}{14cm}{Xccc}
31 \hfill 3 & 2 & 1 & 0 \\
\hline
\multicolumn{1}{|c|}{Reserved} &
\multicolumn{1}{|c|}{UARTINTX} &
\multicolumn{1}{|c|}{UARTTXR} &
\multicolumn{1}{|c|}{UARTRXR} \\
\hline
\end{tabularx}
\caption{UARTSTATUS register}
\end{center}
\end{table}

\begin{description}
\item{1 - UARTINTX\footnote{Introduced in 1.0.1 version}} \hfill \\
UART in TX flag. Reads as 1 if the UART is still transmitting any data, 0 otherwise. This bit should be used to wait until
UART has finished transmitting all data, for example when switching the baud rate.
\item{1 - UARTTXR} \hfill \\
UART TX Ready bit. Reads as 1 when there's space in TX FIFO for transmission, 0 otherwise. This bit should be checked before 
attempting transmission over the UART.

\item{0 - UARTRXR} \hfill \\
UART RX Ready bit. Reads as 1 when there's received data in FIFO, 0 otherwise.

\end{description}



\subsection{UART Prescaler}\label{UART Prescaler}
The UART Prescaler divides system clock and provides baud rate generation for RX and TX. 
An additional prescaler of 16 is used by the RX module. Prescaler can be configured by
setting the appropriate bits in the $UARTCTL$ register.

\begin{displaymath}
Prescale = \frac{f_{OSC}}{baudrate \times 16} - 1
\end{displaymath}

\subsection{Software}
Acessing the UART in software can be done using the $HardwareSerial$ class. The default UART module is 
already instantiated as $Serial$.
\subsubsection{Examples}
The following example uses the default $Serial$ object.
\begin{lstlisting}[language=C++]
void setup()
{
    Serial.begin(115200);              /* Set up serial at 115200 baud */
    Serial.println("Hello world!");    /* Write Hello World! to serial */
}
void loop()
{
    if (Serial.available()) {          /* If there is data on the UART... */
        Serial.write( Serial.read() ); /* ... write it back! */
    }
}
\end{lstlisting}

The follwing example uses a custom UART on IOSLOT 9:

\begin{lstlisting}[language=C++]

HardwareSerial mySerial(9); /* New Serial UART on IO SLOT 9 */

void setup()
{
    mySerial.begin(115200);            /* Set up serial at 115200 baud */
    mySerial.println("Hello world!");  /* Write Hello World! to serial */
}
void loop()
{
    if (mySerial.available()) {          /* If there is data on the UART... */
        mySerial.write( Serial.read() ); /* ... write it back! */
    }
}
\end{lstlisting}
