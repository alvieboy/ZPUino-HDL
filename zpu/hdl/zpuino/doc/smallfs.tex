\section{SmallFS library}

\subsection{What is SmallFS ?}
SmallFS is a read-only, small overhead filesystem intented for
embedded systems, and suitable for use on SPI flash devices.\\
The basic code size is around 900 bytes on ZPUino.


\subsection{Advantages and disadvantages}

Advantages:
\begin{itemize}
\item Position-independent location. Can be put on any stream-capable device,
at any position. Can also be used with memory-mapped systems.

\item Very low overhead, compared to other filesystems like FAT16.
\item Very easy to use.
\item Optionally aligned to any size.

\end{itemize}

Disadvantages:

\begin{itemize}
\item Read-only. You need to regenerate filesystem and fully reprogram it (like 
you do with CDROM filesystems.
\item No directory support, only plain files.
\item 2GiB limit on file size, and on filesystem size.
\item 256-char limit on file names.
\end{itemize}

\subsection{Technical details}

SmallFS uses a very simple layout. The filesystem is composed of an header,
a central directory, and the files themselves. \\
All fields on SmallFS are big-endian. \\

\begin{table}[H]
\begin{center}
\begin{tabularx}{14cm}{cXX}
\hline
\multicolumn{1}{|c|}{Header} &
\multicolumn{1}{|X|}{Directory entries} &
\multicolumn{1}{|X|}{Files} \\
\hline
\end{tabularx}
\caption{SmallFS structure}
\end{center}
\end{table}

\subsubsection{SmallFS header}

\begin{lstlisting}[language=C++]
#define SMALLFS_MAGIC 0x50411F50
struct smallfs_header {
  uint32_t magic;
  uint32_t numfiles;
};
\end{lstlisting}

SmallFS header is 8-bytes long and includes the filesystem magic and the number of
files packed in the filesystem.

\subsubsection{The directory}

Each directory entry is variable sized, and includes the header plus the file name. The file name is not
NULL terminated.

\begin{lstlisting}[language=C++]
struct smallfs_entry {
  uint32_t offset;
  uint32_t size;
  uint8_t namesize;
  char name[0];
};
\end{lstlisting}

The $offset$ field depicts the file contents offset relative to the start of the filesystem.
Following the header, the file name will be stored, with size given by the $namesize$ field.

\subsubsection{The file contents}

All files will be appended after the directory entries. You should use $offset$ field (and add it to 
filesystem offset) to find out where files are stored.

\subsection{The library}

A ready-to-use library is provided to integrate SmallFS in ZPUino.

\begin{lstlisting}[language=C++]
int SmallFS::begin();
SmallFSFile SmallFS::open(const char *name);
\end{lstlisting}

These two methods are the only one you need to use on the base class.\\
The $SmallFS.begin()$ method returns 0 on success, or
-1 otherwise. It only fails if no filesystem is found.\\

In order to open a file, just call the $open()$ method.

\begin{lstlisting}[language=C++]
if (SmallFS.begin()==0) {
  SmallFSFile myfile = SmallFS.open("myfilename.ext");
}
\end{lstlisting}
The $SmallFSFile$ class interface is described below:


\begin{lstlisting}[language=C++]
bool SmallFSFile::valid();
int  SmallFSFile::read(void *buf, int size);
void SmallFSFile::seek(int pos, int whence);
int  SmallFSFile::size() const;
int  SmallFSFile::readCallback(int s,void(*callback)(unsigned char,void*),void *data);
\end{lstlisting}

\subsection{Tools}
A set of tools are provided to create the filesystem, and to extract and/or dump its contents.

\subsection{Integration}
The current IDE (Alpha5+ and beyond) can detect SmallFS and generate automatically its contents and program it after
the sketch binary.\\
All you have to do is to create a directory in your sketch folder with name $smallfs$, put your files
in it, and the IDE will recognise it and program the flash accordingly.


