
\chapter{Peripheral Pin Select}
This information relates to ZPUino 1.0 release. Other releases might have different specs. \\


ZPUino includes a feature which is called Peripheral Pin Select (in short, PPS). 
PPS allows you to map every device input or output pin (such as SPI clock and SPI data lines) to 
each individual pin (GPIO), thus not requiring you to perform synthesis and P\&R each time you want to use
a device on a different IO pin. \\
To simplify things, three methods are supplied to manipulate PPS:

\begin{lstlisting}[language=C++]
void pinModePPS(int pin, int value);
void outputPinForFunction(unsigned int pin,unsigned int function);
void inputPinForFunction(unsigned int pin,unsigned int function);
\end{lstlisting}

Three register blocks exist to configure how pin selection is done. These are called $GPIOPPSIN$,
$GPIOPPSOUT$ and $GPIOPPSMAP$. The above functions manipulate these registers.


\section{Redirecting output}
In order to direct any peripheral output to a GPIO pin, you have to:

\begin{itemize}
\item Configure the GPIO pin as output;
\item Enable PPS on selected GPIO pin;
\item Configure $GPIOPPSOUT$ (outputPinForFunction) to the peripheral signal
\end{itemize}

\subsection{Examples}
The following example maps Sigma Delta 1st channel into GPIO pin number 30:

\begin{lstlisting}[language=C++]
void setup(void)
{
  // Configure pin as output
  pinMode(30,OUTPUT); 
  // enable PPS on this pin
  pinModePPS(30,HIGH); 
  // Map SigmaDelta channel 1 to pin 30
  outputPinForFunction(30, IOPIN_SIGMADELTA1); 
}
\end{lstlisting}
Note that you can use GPIO aliases for your board instead of GPIO number. See your board specific documentation.

\section{Redirecting input}
In order to direct GPIO input into any peripheral, you have to:
\begin{itemize}
\item Configure the GPIO pin as input;
\item Configure $GPIOPPSIN$ (inputPinForFunction) to the peripheral signal;
\end{itemize}

Note that for input you don't need to enable PPS on the pin using $pinModePPS$.\\

\subsection{Examples}
The following example maps USPI MISO signal (Master-In Slave-out) to GPIO pin number 10:
\begin{lstlisting}[language=C++]
void setup(void)
{
  // Configure pin as input
  pinMode(10,INPUT); 
  // Map pin 30 to USPI MISO
  inputPinForFunction(10, IOPIN_USPI_MISO); 
}
\end{lstlisting}
