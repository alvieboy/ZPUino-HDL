\section{GPIO}
\subsection{HDL sources and modules}

\subsubsection{HDL instantiation template}
\begin{lstlisting}[language=VHDL]
component zpuino_gpio is
  generic (
    gpio_count: integer := 32
  );
  port (
    wb_clk_i: in std_logic;
    wb_rst_i: in std_logic;
    wb_dat_o: out std_logic_vector(wordSize-1 downto 0);
    wb_dat_i: in std_logic_vector(wordSize-1 downto 0);
    wb_adr_i: in std_logic_vector(maxIObit downto minIObit);
    wb_we_i:  in std_logic;
    wb_cyc_i: in std_logic;
    wb_stb_i: in std_logic;
    wb_ack_o: out std_logic;
    wb_inta_o:out std_logic;

    spp_data: in std_logic_vector(gpio_count-1 downto 0);
    spp_read: out std_logic_vector(gpio_count-1 downto 0);

    gpio_o:   out std_logic_vector(gpio_count-1 downto 0);
    gpio_t:   out std_logic_vector(gpio_count-1 downto 0);
    gpio_i:   in std_logic_vector(gpio_count-1 downto 0);

    spp_cap_in:  in std_logic_vector(gpio_count-1 downto 0);
    spp_cap_out:  in std_logic_vector(gpio_count-1 downto 0)
  );
end component;
\end{lstlisting}
\subsubsection{Compliance}
The GPIO module is wishbone compatible, in non-pipelined mode.
\subsubsection{Generics}

\begin{description}
\item{gpio\_count} \hfill \\ 
Number of GPIO (1 to 128)
\end{description}

\subsubsection{Source files}

All GPIO system is implemented in \emph{zpuino\_gpio.vhd}.

\subsubsection{PPS capability}

Some pins can have their PPS capability disabled in order to reduce complexity in some designs. The two inputs $spp\_cap\_in$ and
$spp\_cap\_out$ can be manipulated so that some pins have or not in/out PPS capabilities.

\subsection{Location}
GPIO are located in IOSLOT 2.

\subsection{Registers}

Two bits can be manipulated for each GPIO pin, a tristate bit and an output bit. GPIO uses up to 4 32-bit individual
registers in order to map all 128-bits. Note that the actual number of GPIO available is board dependant.

\subsubsection{GPIODATA register}
GPIODATA is actually a 4-word register. Each bit maps a single GPIO input/output. If the GPIO is configured as an output (see 
$GPIOTRIS$ register for input/output selection) writing a 0 or 1 to a specific bit will cause the underlying pin to become 0 or 1 
(except if pin is currently mapped to a peripheral - see PPS for more details). When read it will depict the values at an input (if
the pin is configured as an input) or the current output value if it's configured as an output.
\begin{table}[H]
\begin{center}
\begin{tabularx}{14cm}{X}
31 \hfill 0 \\
\hline
\multicolumn{1}{|c|}{\tiny GPIODATA [31:0]} \\
\hline
63 \hfill 32 \\
\hline
\multicolumn{1}{|c|}{\tiny GPIODATA [63:32]} \\
\hline
95 \hfill 64 \\
\hline
\multicolumn{1}{|c|}{\tiny GPIODATA [95:64]} \\
\hline
127 \hfill 96 \\
\hline
\multicolumn{1}{|c|}{\tiny GPIODATA [127:96]} \\
\hline
\end{tabularx}
\caption{GPIODATA register}
\end{center}
\end{table}

\subsubsection{GPIOTRIS register}
$GPIOTRIS$ is actually a 4-word register. When a bit is set to one, the GPIO becomes an input (it will be tristated).

\begin{table}[H]
\begin{center}
\begin{tabularx}{14cm}{X}
31 \hfill 0 \\
\hline
\multicolumn{1}{|c|}{\tiny GPIOTRIS [31:0]} \\
\hline
63 \hfill 32 \\
\hline
\multicolumn{1}{|c|}{\tiny GPIOTRIS [63:32]} \\
\hline
95 \hfill 64 \\
\hline
\multicolumn{1}{|c|}{\tiny GPIOTRIS [95:64]} \\
\hline
127 \hfill 96 \\
\hline
\multicolumn{1}{|c|}{\tiny GPIOTRIS [127:96]} \\
\hline
\end{tabularx}
\caption{GPIOTRIS register}
\end{center}
\end{table}



\subsection{Software}

GPIO pins should be manipulated using the Arduino-like functions $digitalWrite()$, $digitalRead()$ and $pinMode()$.
Additionaly GPIO can be bound to devices using PPS (Peripheral Pin Select).
                                   
You can also use the following "C" macros if needed. Each pin block is 32-bit wide:

\begin{description}
\item{GPIODATA(x)} \hfill \\ GPIODATA for pin block x
\item{GPIOTRIS(x)} \hfill \\ GPIOTRIS for pin block x
\end{description}
