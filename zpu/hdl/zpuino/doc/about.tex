\chapter{ZPUino}
ZPUino is a SoC (System-on-a-Chip) based on Zylin's ZPU 32-bit processor core.

\subsection{What is inside ZPUino}

ZPUino is a very modular system, and it can integrate a whole set of devices.
On the hardware side, a stock ZPUino image usually contains:

\begin{itemize}
\item ZPU Extreme Core, a modified ZPU core
\item One or two UART interfaces
\item Two SPI interfaces
\item A 16-bit and a 24-bit timer.
\item One TSC (Time Stamp Counter)
\item 128-bit GPIO interface
\item System Controller
\item Two SigmaDelta outputs
\item Peripheral Pin Select
\item Optional VGA and/or HDMI outputs.
\end{itemize}

See chapter \ref{iodevices} which discuss IO devices for more information.

Software-wise, it supports the following features:

\begin{itemize}
\item 4Kb Bootloader, which includes required emulation code for ZPU.
\item Bootstraps code from program flash (shadows into RAM)
\item Serial programming of program flash.
\end{itemize}

\subsection{Where does it run}
First implementation was done on Spartan3E 500 (-4), on a S3E Starter Kit, with a M25P16 SPI flash ROM and 32Kbytes RAM.\\
Right now it runs on other many boards, see chapter \ref{boards} for more information. \\
unning speed: up to 100MHz, 96MHz recommended.

\subsection{IO Device features}
Main features:
\begin{itemize}
\item All IO Devices should be wishbone compatible.
\item IO Devices are placed in slots. A total of 15 slot exist, and can be extended using hubs.
\item All devices have an ID, and their mapping can be automatically detected in software. Anyone can request a Vendor ID from
ZPUino project for his own devices.
\end{itemize}


UART features:
\begin{itemize}
\item 16-bit prescaler
\item 2048-byte deep receive FIFO. (16 byte in reduced implementations)
\end{itemize}

SPI features:
\begin{itemize}
\item Programmable prescaler
\item Configurable CPOL
\item Configurable SRE (Sample on Rising Edge)
\item 8, 16, 24 and 32-bit transmission modes, 4 byte (32 bit) receive register.
\item Configurable blocking operation
\end{itemize}

Timer features:
\begin{itemize}
\item Two independent timers
\item 10-bit prescaler (or no prescaler)
\item 16-bit and 24-bit counter
\item Count-up and Count-down mode
\item 16-bit Compare register
\item Clear on Compare match support
\item Interrupt support on Compare Match
\item Output compare Registers to GPIO pin (allows PWM)
\end{itemize}

GPIO features:
\begin{itemize}
\item Up to 128 GPIO entries
\item Bi-directional (tristate) configuration
\end{itemize}

SigmaDelta features:
\begin{itemize}
\item Two 16-bit channel
\item Runs at system clock speed
\item Configurable endianness and signed/unsigned.
\item Based in OpenCores SigmaDelta.
\end{itemize}

PPS (Peripheral pin select) features:
\begin{itemize}
\item Map every device pin to any GPIO pin
\item Fully configurable in run-time
\end{itemize}

Main VGA/HDMI features:
\begin{itemize}
\item Generic VGA up to 2048x2048 (including blanking, software controlled).
\item Configurable bit depth (RGB332 or RGB565)
\item Configurable H/V duplication for lower-resolution modes.
\end{itemize}

ZPUino programmer:
\begin{itemize}
\item Up to 1Mbps programming speed
\item Supports programming both sketches and FPGA bitfile [if ZPUino bootloader already loaded]
\item Support for upload-to-ram
\end{itemize}

\subsection{Hardware support}

Currently ZPUino should run on any Spartan3E FPGA (from 500 and beyond), and Spartan6 FPGA (LX4 and beyond). \\
List of flash devices currently supported by programmer:
\begin{itemize}
\item ST M25P16
\item ST M25P32
\item MICRON N25Q128A
\item SST SST25VF040B
\item ATMEL AT45DB081D
\item ATMEL AT45DB081D
\item MACRONIX MX25L6445E
\item WINBOND W25Q80BV
\item SPANSION S25FL064P
\end{itemize}
Other flash might be supported. Specifically flash from same series, but with different capacities, can be easily added to programmer.
For other flash vendors or models contact us by e-mail. \\
External memory modules are supported, such as SRAM, SDRAM and DDR SDRAM.\\
For more information on which modules are currently tested, contact us.

